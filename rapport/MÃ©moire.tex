% !TeX spellcheck = fr_FR
\documentclass[a4paper]{article} 

\usepackage[utf8]{inputenc}
\usepackage[french]{babel}
\usepackage{graphicx}
\usepackage{url}
\usepackage[toc,page]{appendix}
\usepackage{bm}
\renewcommand{\appendixpagename}{Annexes}
\renewcommand{\appendixtocname}{Annexes}

\author{Raphael Palmeri\\1 ère Master en Sciences Informatiques\\ 2020-2021}
\begin{document}

\begin{titlepage}
	\centering
	{\scshape\LARGE Université de Mons \par}
	\vspace{1cm}
	{\scshape\large Faculté des Sciences \par}
	\vspace{1.5cm}
	{\huge\bfseries Le phénomène de double descente au sein de l'Apprentissage automatique \par}
	\vspace{2cm}
	{\Large\itshape Raphael Palmeri \par}
	\vspace{2.0cm}
	{Sous la direction de Souhaib Ben Taieb \par}
	\vspace{1.0cm}
	{Master en Sciences informatiques \par}
	\vspace{0.5cm}
	{Je soussigné, Palmeri Raphael, atteste avoir respecté les règles éthiques en vigueur \par}
	\vspace{4cm}
	Année universitaire 2020-2021
	\vfill
	\includegraphics[scale=0.3]{"Faculte_Sciences_logo".png}
	\hfill
	\includegraphics[scale=0.4]{"UMONS-Logo".jpg}
\end{titlepage}
\newpage
\thispagestyle{empty}
\mbox{}
\newpage

\tableofcontents
\newpage

\section{Introduction}
Dans le cadre du Master en Sciences Informatiques en horaire décalé, il est demandé de réaliser un mémoire sur un sujet proposé par les enseignants de l'UMons de la section Sciences Informatiques. Mon choix de sujet s'est porté sur "Le phénomène de double descente au sein de l'Apprentissage Machine (\textit{'The double descent phenomenon in machine learning'})".\newline

Dans un premier temps, je vais présenter ce qu'est l'apprentissage Machine ainsi qu'expliquer brièvement les différentes sous-catégories existantes au sein de celui-ci. \newline

Dans une deuxième partie, je vais expliquer les notions de biais et de variance, ainsi que le phénomène de compromis entre ces deux notions au sein de l'apprentissage machine. \newline

Dans la troisième partie, je vais démontrer mathématiquement le compromis biais-variance et expliquer cette démonstration plus en détails. \newline

Dans la quatrième partie, à l'aide de simulations, nous montrerons l'effet de ce phénomène ainsi que son comportement en face aux modifications des différents paramètres de la simulation.\newline

\newpage

\section{Apprentissage automatique ("Machine Learning")}

L'apprentissage automatique est la capacité d'un ordinateur à "apprendre" en se basant sur des données mises à sa disposition. Le terme "apprendre" désigne la capacité à détecter/trouver des répétitions (\textit{'patterns'}) dans ces données. Ces répétitions permettent à la machine de donner une expertise par rapport à un problème donné ou une réponse à une certaine question à l'aide d'un modèle qu'il aura construit lors de cette apprentissage.\cite{MLPracticalApproach}\cite{UnderstandingML} \newline 

L'apprentissage automatique est notamment utilisé dans différents domaines tels que l'automobile avec ses voitures sans conducteurs, les finances afin de notamment détecter les fraudes, mais aussi le domaine de la santé avec la possibilité d'essayer de produire un diagnostic sur base des informations disponibles à propos d'un patient, etc. \newline

Afin de réaliser de l'apprentissage automatique, il est nécessaire d'avoir deux choses: un algorithme d'apprentissage (voir sous-section \ref{LearningAlgo}) et des données (voir sous-section \ref{Data}) que nous allons explorer dans les prochaines sections.

\newpage

\subsection{Algorithme d'apprentissage}
\label{LearningAlgo}
Il existe différentes catégories d'apprentissages en machine learning:

\subsubsection{Apprentissage supervisés}
Dans ce type d'apprentissage, la machine reçoit un ensemble de données avec les classes de tout les exemples existants.
Par exemple, un expert aura déjà défini les différentes classes possibles pour la reconnaissance d'objets via des images ('Chaise', 'Table', 'Chien', 'Chat', ...). Les algorithmes de cette famille vont dès lors se baser sur ces classes déjà définies afin de pouvoir attribuer une classe (que l'on espère correcte) à une nouvelle donnée encore inconnue. Dans l'exemple ci-dessus, on parlera de \textbf{"Classification"}. \newline
Il existe aussi des problèmes de \textbf{"Régression"}, ceux-ci tentent de lier une nouvelle donnée à un nombre réel. Par exemple, dans le cadre de l'estimation de prix de maison. \newline

Étant donné le fait que le compromis Biais-Variance (voir section \ref{B-V}) n'existent qu'au sein de cette catégorie d'algorithmes, il est logique que celle-ci soit la famille que nous étudierons le plus dans ce rapport.

\subsubsection{Apprentissage non-supervisés}
Dans le cas d'algorithme non-supervisé, les données d'entrainement et de test sont mélangés. Le modèle n'aura aucun exemple pour s'aider à détecter un pattern, il devra le faire par lui même en étudiant les similarités entre les différentes données. Ces dernières seront ensuite rangées par groupes afin qu'un expert puisse par exemple les utiliser pour une recherche scientifique. Dans le cadre de l'utilisation de cette famille, on parlera de \textbf{"Clustering"}. \newline

\subsubsection{Apprentissage semi-supervisés et Apprentissage actif}
Dans la plupart des situations, il est impossible de classifier l'ensemble des données d'apprentissage. Dans ce genre de cas, la machine doit dès lors apprendre des classes qui lui sont fournies, mais aussi des données non labellisées, c'est ce que l'on appelle l'apprentissage semi-supervisé. Dans le cadre où ce n'est pas un expert qui donne les classes mais bien la machine qui tente de leur donner un label, on se trouve dans le cas de l'apprentissage actif. 

\subsubsection{Apprentissage par transfert et apprentissage multitâche}
L'idée principale derrière l'apprentissage par transfert est d'aider le modèle à s'adapter à des situations qu'il n'as pas rencontrés précédemment. Cette forme d'apprentissage s'appuie sur le fait d'effectuer des réglages précis sur un modèle générale qui lui permettra de travailler dans un nouveau domaine.

\newpage

\subsubsection{Apprentissage par renforcement}
L'apprentissage par renforcement se base sur l'idée de maximiser une récompense selon une ou plusieurs actions. On va dés lors définir en fonction des actions, si elles sont encouragées, ou au contraire, découragées.

\subsubsection{Exemples d'algorithmes}

\begin{itemize}
	\item Prédicteurs linéaires (\textit{'Linear Predictors'}) : tels que la régression linéaire, perceptron, etc.
	\item Boosting
	\item Support Vector Machines
	\item Arbres de décision (\textit{'Decision Trees'})
	\item Voisin le plus proche (\textit{'Nearest Neighbor'})
	\item Réseau de neurones (\textit{'Neural Networks'})
	\item ...
\end{itemize}

\newpage

\subsection{Notation mathématiques}
Dans cette sous-section, je vais expliquer les différentes notations mathématiques nécessaires à la bonne compréhension des prochains chapitres. \newline

Dans le cadre d'un apprentissage automatique supervisé, on cherche à prédire un résultat $y\in \bm{Y}$ à partir d'une donnée $x \in \bm{X}$ où les paires $(x,y)$ proviennent d'une distribution inconnue $\bm{D}$. \newline

Le problème d'apprentissage automatique consiste à apprendre une fonction $f' : \bm{X} \to \bm{Y}$ à partir d'un ensemble de données d'entrainement fini $\bm{S}$ contenant $m$ variables indépendantes et identiquement distribuées (\textit{i.i.d}) provenant de $\bm{D}$. \newline

$f'$ peut aussi être vue comme étant une hypothèse $h \in \bm{H}$, choisie à partir d'une classe d'hypothèses $\bm{H$} contenant des fonctions possibles pour le modèle. \newline

Dans un cadre idéal, la fonction $f'$ serait équivalente à la fonction $f$, la 'vrai' fonction qui régit $\bm{X} \to \bm{Y}$. \newline

Pour une fonction de perte $l: \bm{Y} \times \bm{Y}$, la qualité d'un prédicateur $h$ peut être quantifié par le \textit{risque} (ou \textit{l'erreur attendue}):

\[ R(h) = \bm{E}_{(x,y)\sim D} \bm{l}(h(x),y) \]

Le but de l'apprentissage supervisé est de trouver $\min_{h \in \bm{H}} R(h)$, c'est-à-dire la valeur minimal de l'erreur attendue. Malheureusement, il nous est impossible de calculer réellement cette valeur car nous ne connaissons pas $\bm{D}$. Ce que nous connaissons, par contre, c'est \textit{l'erreur d'entrainement} :

\[ \hat{R}(h) = \bm{E}_{(x,y)\sim S} \bm{l}(h(x),y) \]

Cette erreur est issue de $\bm{S}$ qui est un ensemble provenant de $\bm{D}$, nous pouvons dés lors tenter de trouver la valeur minimal de celle-ci comme étant un substitut de celle de $R(h)$.

\subsubsection{La fonction de perte quadratique}
Elle s'exprime comme suit :

\[ (y - y')^2 \]

où \textbf{y} représente la valeur véritable de la vrai fonction et \textbf{y'} représente la valeur estimée par le modèle. \newline

L'erreur quadratique moyenne quant à elle n'est que la moyenne des erreurs sur l'ensemble des données :

\[ MSE = \frac{1}{n} \sum_{(x,y)\in D} (y - y')^2 \]

\subsubsection{Espérance mathématique d'une variable aléatoire}
Dans les preuves de la décomposition du biais-variance, on peut y trouver une notation statistique appelée l'espérance mathématique qui se note $ E(x) $ pour une variable aléatoire $x$. \newline

L'espérance mathématique représente la moyenne pondérée des valeurs que peut prendre cette variable. 

\newpage

\subsection{Données}
\label{Data}
Afin de permettre le bon fonctionnement de l'apprentissage automatique, il est nécessaire d'avoir des données. Celles-ci doivent être présentes en quantité et, dans le meilleur des cas, elles doivent êtres "nettoyées" c'est-à-dire qu'il faut parfois retirer des attributs inutiles, en modifier certains pour qu'il soit compréhensibles pour l'algorithme. En effet, il arrive parfois qu'un attribut défini dans les données ne soit pas exploitables par l'algorithme sans être modifiés ou précisés. \newline

Ces données peuvent être distinguées en 2 catégories:

\subsubsection{Données d'apprentissage}
Ces données sont des exemples déjà traités par un expert dans le domaine. Elles peuvent être utilisés comme exemple d'apprentissage pour les algorithmes supervisés. Grâce à celles-ci, l'algorithme pourra générer un modèle qui pourra estimer la valeur (ou la classe ) en fonction d'une donnée inconnue.

\subsubsection{Données de test}
Ces données sont destinés à valider le modèle créé par l'algorithme d'apprentissage. l'idée est de fournir des données inconnues du modèle, afin de vérifier et valider son comportement. Si le modèle produit des résultats extrêmement éloignés de la vérité, c'est qu'il n'est pas encore prêt. Il faut donc repasser par une phase d'apprentissage en fournissant potentiellement plus de données d'apprentissage et/ou en les rendant plus précises afin que la machine établisse un nouveau modèle dont les réponses seront plus correctes.

\newpage

\section{Le compromis biais - variance}
\label{B-V}

Afin de mieux comprendre la notion du compromis de Biais-Variance, il est important de comprendre sa version plus générale qui est le compromis Approximation-Estimation.

\subsection{Le compromis approximation - estimation}
\label{A-E}

Le compromis approximation-estimation influencera l'erreur que réalisera un modèle sur ses prédictions. C'est un composant de l'erreur totale au sein de l'apprentissage machine. C'est aussi le seul élément que l'on peut tenter de minimiser afin de limiter celle-ci. \newline


L'erreur totale est constitué de 3 choses: \newline

\[ error_{total} = error_{generalization} + error_{training} + error_{irreductible} \]

1. l'erreur de généralisation ou d'approximation (\textit{'generalization error'}): cette erreur est la conséquence de la sélection d'un sous-ensemble que l'on considère comme étant représentatif. De part cette sélection, on induit une possible erreur.\newline

2. l'erreur d'entrainement ou d'estimation (\textit{'training error'}) : cette erreur est la conséquence de l'apprentissage. Dans nos données sélectionnées pour l'apprentissage de la machine, on peut avoir des cas spécifiques qui ne se présentent que dans notre ensemble d'apprentissage, ce qui mènera le modèle à un 'biais d'apprentissage' et peut diminuer la précision de celui-ci lors de l'utilisation de données de test.\newline

3. l'erreur irréductible (\textit{'irreductible error'}): cette erreur est la conséquence d'un traitement peu efficace des données en amont de l'apprentissage. Les données qui seront utilisées par l'algorithme doivent êtres nettoyées avant d'être utilisées. Cette erreur ne dépends donc pas de l'algorithme directement.\newline
 \textit{"Garbage In, Garbage Out"}, ce qui signifie que si les données en entrée ne sont pas correctes, les résultats ne sauraient l'être. \newline

\newpage

\subsection{Le compromis biais - variance}

Le compromis biais-variance permet de quantifier le compromis approximation-estimation lorsque l'on utilise la fonction de perte quadratique (ou perte \textit{L2}) et plus particulièrement, l'erreur quadratique moyenne (\textit{MSE}).

\subsubsection{Le biais}
Le biais est la mesure qui montre à quel point le modèle établi par l'algorithme d'apprentissage supervisé est proche de la 'vrai' fonction d'un problème donné. \newline

La figure \ref{BiasRepresentation} montre une représentation graphique du biais. \textit{f} représente la 'vrai' fonction d'un problème donné, \textit{H} représente l'ensemble des hypothèses choisies et le point noir représente une hypothèse sélectionnée parmi \textit{H}. \newline

\begin{figure}[!h]
	\centering
	\includegraphics[scale=1]{"Representation Biais".png}
	\caption{Représentation du Biais.}
	\cite{BiasVarianceTradeoffTextbooksUpdate}
	\label{BiasRepresentation}
\end{figure}

\subsubsection{La variance}

La variance est la variation entre la valeur d'un ensemble de données de test par rapport à la valeur donnée par le modèle choisi. 

\newpage

\begin{figure}[!h]
	\centering
	\includegraphics[scale=0.75]{"BiasVarianceTradeoff".png}
	\caption{Représentation de l'erreur en fonction de la complexité du modèle}
	\cite{BiasVarianceTradeoffTextbooksUpdate}
	\label{TradeoffRepresentation}
\end{figure}

La figure \ref{TradeoffRepresentation} montre l'impact du biais et de la variance sur l'erreur d'un modèle et ce en fonction de sa complexité. Elle montre aussi très bien le point d'optimisation de la complexité du modèle lié au compromis entre le biais et la variance.

\newpage

Un exemple concret du compromis biais-variance est celui du tir sur cible :

\begin{figure}[!h]
	\centering
	\includegraphics[scale=0.5]{"exemple Bias-Variance".png}
	\caption{Exemple concret du Compromis Biais-Variance.} \cite{UnderstandingBiasVarianceTradeoff}
	\label{ConcreteExample}
\end{figure}

Dans la figure \ref{ConcreteExample}, on a quatre cibles selon deux axes différents, le biais et la variance, chacun de ces axes peut-être soit faible, soit élevé. \par

Le premier cas (biais faible et variance faible) représente un excellent tireur, il vise toujours le centre et ses tirs sont fortement groupés. \par

Le second cas (biais faible et variance élevée) représente un 'bon' tireur, il vise le centre mais ses tirs sont assez dispersés. \par

Le troisième cas (biais élevé et variance faible) représente un tireur moyen, il ne vise pas le centre mais n'est pas non plus hors de la cible ou au bord de celle-ci et ses tirs sont fortement groupés. \par

Le quatrième cas (biais élevé et variance élevée) représente un mauvais tireur, il ne vise pas le centre et ses tirs sont très dispersés. \par

\newpage

\subsection{Décomposition du biais-variance}
\label{decomposition_Biais_Variance_section}

Considérons le modèle suivant : 
\begin{equation}
\label{decomposition_Biais_Variance}
y = f(x) + \epsilon
\end{equation}

où : 

\begin{itemize}
	\item x $\sim$ p(x)
	\item f est une fonction fixée inconnue
	\item $\epsilon$ est du bruit aléatoire tel que :
	\begin{itemize}
		\item $E[\epsilon|x] = 0$
		\item $Var(\epsilon|x) = \sigma^2$
	\end{itemize}
\end{itemize}


étant donné un set de données $D = {(x_i, y_i)}^n_{i=1}$ où $(x_i, y_i)$ est un échantillon provenant de (1), et un ensemble d'hypothèses $H$, on calcule : \newline
\[ g^{(D)} = argmin_{h\in H}  E_{in}(h) := \frac{1}{n} \sum_{i=1}^{n} L(y_i,h(x_i)) \]

étant donné D, l'erreur au carré hors-échantillon de $g^{(D)}$ est : \newline
\[ E_{out}(g^{(D)}) = E_{x,y}[(y - g^{(D)}(x))^2] \]

considérons 
\begin{equation}
E_D[E_{out}(g^{(D)})] = E_{x,y,D}[(y - g^{(D)}(x))^2]
\end{equation}
représentant l'espérance moyenne sur les variables x, y et D. \newline

En prenant $ \bar{g} = E_D[g^{(D)}(x)]$ , on peut décomposer (2) comme suit 
\[ \underbrace{E_x[(f(x) - \bar{g}(x))^2]}_{Biais} + \underbrace{E_{x,D}[(\bar{g}(x) - g^{(D)}(x))^2]}_{Variance} + \underbrace{\sigma^2}_{Variance irreductible} \]

\newpage

On sait que le compromis biais-variance s'exprime comme suit : 
\[ E_{x,y,D}[(y-g^{(D)}(x))^2] \]
En sachant que le g moyen est :
\begin{equation}
\label{g_moyen}
\bar{g}(x) = E_D [g^{(D)}(x)]
\end{equation}
Prouvons qu'il est égal à :
\[ E_x[(f(x) - \bar{g}(x))^2] + E_{x,D}[(\bar{g}(x) - g^{(D)}(x))^2] + \sigma^2 \]
(Voir Preuve en annexe \ref{MathematicalProof})

\newpage

\section{Simulation}
Afin de montrer le compromis biais-variance, j'ai réalise une série de simulations.

\subsection{Simulation sur le nombre d'échantillons}
Dans cette sous-section, les simulations ont été réalisées sur la base d'un changement de nombre d'échantillons utilisés, sélectionnés aléatoirement, comme données d'entrainement par l'algorithme pour définir un modèle. Afin d'obtenir une moyenne d'erreur correcte, chaque algorithme crée 20 modèles différents. \newline

Les différentes tailles d'échantillons utilisées sont : $100, 500, 1000, 2000, 4000, 8000, 10000$. \newline

Les données sont basés sur un ensemble de données représentant les propriétés physico-chimiques de la structure tertiaire des protéines (disponible à l'adresse suivante : \url{https://archive.ics.uci.edu/ml/datasets/Physicochemical+Properties+of+Protein+Tertiary+Structure}). Aucune traitement spécifiques n'a été fait sur les données, elle sont utilisés telles quelles. \newline

Les données de test sont un échantillon de 1000 enregistrements choisi aléatoirement sur les 45730 disponibles dans l'ensemble. \newline 

Le code utilisé est présenté dans l'annexe \ref{SimulationSampleCountCode}.

\newpage

\subsubsection{Régression Linéaire}

\begin{figure}[!h]
	\centering
	\includegraphics[scale=0.35]{"LR_SampleCount_sim".png}
	\caption{Simulation d'une Régression Linéaire en fonction du nombre d'échantillons}
	\label{LR_SampleCount}
\end{figure}

Dans la figure \ref{LR_SampleCount}, on remarque que le biais est faible au départ de la simulation, il a approximativement une valeur de $0.25$ et qu'il tends vers $0$ lorsque le nombre d'échantillons augmente. Cela indique possiblement que l'ensemble de données est régi par une fonction linéaire et que le modèle c'est approché de cette "vraie" fonction.\newline

Pour ce qui est de la variance, elle a tout d'abord une valeur approximativement égale à $1.75$ avant de décroitre rapidement avec l'augmentation du nombres d'échantillons jusqu'à plus ou moins 1000 échantillons, où la diminution se fait de manière plus réduite. Elle finie par tendre vers $0$ lorsque l'on dépasse les 8000 échantillons. \newline

On observe que l'erreur quadratique croit en début de simulation avant d'atteindre son paroxysme à la valeur de $7.2$ lorsque 500 enregistrements sont utilisés. Finalement, elle décroit rapidement entre 500 et 1000 enregistrements avant d'atteindre sa valeur la plus basse avec 8000 enregistrements qui est de $6.4$.

\newpage

\subsubsection{Arbre de décision}

\begin{figure}[!h]
	\centering
	\includegraphics[scale=0.35]{"DT_SampleCount_sim".png}
	\caption{Simulation d'un Arbre de décision en fonction du nombre d'échantillons}
	\label{DT_SampleCount}
\end{figure}

Dans la figure \ref{DT_SampleCount}, on remarque que le biais tends vers $1$ en partant de $1.5$ en début de simulation. Ce n'est à proprement parler pas un mauvais résultat mais en le comparant à la régression linéaire, cette valeur est plus importante. \newline

La variance part de la valeur $6$ en début de simulation, pour ne jamais descendre en dessous $3.5$. En comparatif de la régression linéaire, la valeur de variance est plus importante. On remarque aussi qu'elle n'est pas constamment entrain de décroitre ou de croitre mais qu'elle oscille entre les deux. \newline

L'erreur quadratique oscille entre décroissement et accroissement jusqu'à la valeur clé de 2000 enregistrements où celle-ci ne fait que décroitre pour atteindre son minimum avec 8000 enregistrements avant de croitre à nouveau.

\newpage

\subsubsection{Bagging}

\begin{figure}[!h]
	\centering
	\includegraphics[scale=0.35]{"BAG_SampleCount_sim".png}
	\caption{Simulation d'un Bagging en fonction du nombre d'échantillons}
	\label{BAG_SampleCount}
\end{figure}

Dans la figure \ref{BAG_SampleCount}, on remarque que le biais tends vers $0.5$ en partant de $1.1$ en décroissement tout d'abord rapidement et ensuite de manière moins marqué en arrivant à 500 enregistrements. \newline

La variance part de la valeur $1.6$ en début de simulation, pour ensuite décroitre rapidement jusqu'à 500 enregistrements et finir par décroitre plus lentement à partir de 2000 enregistrements pour finalement atteindre $0.7$ qui est sa valeur minimum. \newline

L'erreur quadratique décroit fortement en début de simulation pour s'accroitre à nouveau après 500 enregistrements. A 4000 enregistrements, l'erreur quadratique ne fait plus que décroitre pour atteindre son minimum en $4.5$.

\newpage

\subsubsection{Random Forest}

\begin{figure}[!h]
	\centering
	\includegraphics[scale=0.35]{"RF_SampleCount_sim".png}
	\caption{Simulation d'une Forêt aléatoire en fonction du nombre d'échantillons}
	\label{RF_SampleCount}
\end{figure}

Dans la figure \ref{RF_SampleCount}, on remarque que le biais et la variance se croisent deux fois lors de la simulation. Le biais décroit rapidement en début de simulation et décroit plus lentement au passage des 2000 enregistrements. Il arrive finalement à son minimum de $0.41$ en fin de simulation. \newline

La variance, elle aussi, décroit rapidement avant les 2000 enregistrements pour finalement décroitre plus lentement. Elle atteint son minimum de $0.39$ en fin de simulation également. \newline

L'erreur quadratique décroit rapidement jusqu'à 1000 enregistrements, s'accroit légèrement jusque 2000 enregistrements pour finalement atteindre son minimum à 8000 enregistrements avant de repartir à la hausse.

\newpage

\subsubsection{Ada Boost}

\begin{figure}[!h]
	\centering
	\includegraphics[scale=0.35]{"ADAB_SampleCount_sim".png}
	\caption{Simulation Ada Boost en fonction du nombre d'échantillons}
	\label{ADAB_SampleCount}
\end{figure}

Dans la figure \ref{ADAB_SampleCount},

\newpage

\subsubsection{K plus proches voisins}

\begin{figure}[!h]
	\centering
	\includegraphics[scale=0.35]{"KNN_SampleCount_sim".png}
	\caption{Simulation K plus proches voisins en fonction du nombre d'échantillons}
	\label{KNN_SampleCount}
\end{figure}

Dans la figure \ref{KNN_SampleCount},

\newpage

\section{Conclusion}

\newpage

\begin{appendices}
	
	\section{Preuve mathématique}
	\label{MathematicalProof}
	
	Reprenons l'expression du compromis :
	\begin{equation}
		\label{Bias_Variance_formula}
		E_{x,y,D}[(y-g^{(D)}(x))^2]
	\end{equation}
	
	On peut exprimer (\ref{Bias_Variance_formula}) comme suit : 
	
	\begin{equation}
		\label{Bias-Variance_formula_proof_1}
		E_x [E_{y,D} [(y - g^{(D)}(x))^2 | x]]
	\end{equation}
	
	En fixant $x$, on peut simplifier (\ref{Bias-Variance_formula_proof_1}) :
	
	\begin{equation}
		\label{Bias-Variance_formula_proof_2}
		E_{y,D}[ (y - g^{(D)}(x))^2]
	\end{equation}
	
	En ajoutant $ - f(x) + f(x) $ à l'équation \ref{Bias-Variance_formula_proof_2}, on obtient :
	
	\begin{equation}
		\label{Bias-Variance_formula_proof_3}
		E_{y,D}[ (y - f(x) + f(x) - g^{(D)}(x))^2]
	\end{equation}
	
	En considérant $ y - f(x) $ comme étant $a$ et $ f(x) - g^{(D)}(x)$ comme $b$ et en appliquant la formule $(a+b)^2 = a^2 + b^2 + 2ab$ dans l'équation (\ref{Bias-Variance_formula_proof_3}), on obtient :
	
	\begin{equation}
		\label{Bias-Variance_formula_proof_4}
		E_{y,D} [(y-f(x))^2] + E_{y,D} [(f(x) - g^{(D)}(x))^2] + 2E_{y,D} [(y-f(x)) (f(x) - g^{(D)}(x)) ]
	\end{equation}
	
	En utilisant (\ref{decomposition_Biais_Variance}) pour remplacer $y$ dans l'équation (\ref{Bias-Variance_formula_proof_4}), on obtient :
	
	\begin{equation}
		\label{Bias-Variance_formula_proof_5}
		E_{y,D} [(f(x) + \epsilon - f(x))^2] + E_{y,D} [(f(x) - g^{(D)}(x))^2] + 2E_{y,D} [(y-f(x)) (f(x) - g^{(D)}(x)) ]
	\end{equation}
	
	En simplifiant l'équation (\ref{Bias-Variance_formula_proof_5}), on obtient :
	
	\begin{equation}
		\label{Bias-Variance_formula_proof_6}
		E_{y,D} [(\epsilon)^2] + E_{y,D} [(f(x) - g^{(D)}(x))^2] + 2E_{y,D} [(y-f(x)) (f(x) - g^{(D)}(x)) ]
	\end{equation}
	
	En utilisant la définition de la Variance de $\epsilon$ de la section \ref{decomposition_Biais_Variance_section}, on peut simplifier l'équation (\ref{Bias-Variance_formula_proof_6}) comme suit :
	
	\begin{equation}
		\label{Bias-Variance_formula_proof_7}
		\sigma^2 + E_{y,D} [(f(x) - g^{(D)}(x))^2] + 2E_{y,D} [(y-f(x)) (f(x) - g^{(D)}(x)) ]
	\end{equation}
	
	En utilisant (\ref{intermediate_proof_a_9}) (voir section \ref{intermediate_proof_a_subsection}) dans l'équation (\ref{Bias-Variance_formula_proof_7}), on obtient :
	
	\begin{equation}
		\label{Bias-Variance_formula_proof_8}
		\sigma^2 + E_{y,D} [(f(x) - g^{(D)}(x))^2] + 0
	\end{equation}
	
	En ajoutant $ -\bar{g}(x) + \bar{g}(x)$ à l'équation (\ref{Bias-Variance_formula_proof_8}) dans le terme $ E_{y,D} [(f(x) - g^{(D)}(x))^2]$, on obtient :
	
	\begin{equation}
		\label{Bias-Variance_formula_proof_9}
		\sigma^2 + E_{y,D} [(f(x) -\bar{g}(x) + \bar{g}(x) - g^{(D)}(x))^2] + 0
	\end{equation}
	
	En considérant $ f(x) -\bar{g}(x) $ comme étant $a$ et $ \bar{g}(x) - g^{(D)}(x)$ comme $b$ et en appliquant la formule $(a+b)^2 = a^2 + b^2 + 2ab$ dans l'équation (\ref{Bias-Variance_formula_proof_9}), on obtient :
	
	\begin{equation}
		\label{Bias-Variance_formula_proof_10}
		\sigma^2 + E_{y,D} [(f(x) -\bar{g}(x))^2] + E_{y,D} [(\bar{g}(x) - g^{(D)}(x))^2] + 2 E_{y,D} [(f(x) -\bar{g}(x)) (\bar{g}(x) - g^(D)(x))]
	\end{equation}
	
	En vérifiant les espérances, on peut encore simplifier l'équation (\ref{Bias-Variance_formula_proof_10}) en :
	
	\begin{equation}
		\label{Bias-Variance_formula_proof_11}
		\sigma^2 + (f(x) -\bar{g}(x))^2 + E_{D} [(\bar{g}(x) - g^{(D)}(x))^2] + 2 E_{y,D} [(f(x) -\bar{g}(x)) (\bar{g}(x) - g^(D)(x))]
	\end{equation}
	
	En utilisant la preuve intermédiaire (\ref{intermediate_proof_b_5}) (voir section \ref{intermediate_proof_b_subsection}), on obtient l'équation suivante :
	
	\begin{equation}
		\label{Bias-Variance_formula_proof_12}
		\sigma^2 + (f(x) -\bar{g}(x))^2 + E_{D} [(\bar{g}(x) - g^{(D)}(x))^2] + 0
	\end{equation}
	
	et finalement en ré-appliquant l'espérance de x que nous avions retiré pour faciliter la notation, on obtient :
	
	\begin{equation}
		\label{Bias-Variance_formula_proof_13}
		\sigma^2 + E_x[(f(x) -\bar{g}(x))^2] + E_{x,D} [(\bar{g}(x) - g^{(D)}(x))^2]
	\end{equation}
	
	ce qui prouve bien que $ E_{x,y,D}[(y-g^{(D)}(x))^2] $ est équivalent à (\ref{Bias-Variance_formula_proof_13})
	\newpage
	
	\section{Preuve intermédiaire de $2E_{y,D} [(y-f(x)) (f(x) - g^{(D)}(x)) ]$}
	\label{intermediate_proof_a_subsection}
	Prouvons que $E_{y,D} [(y-f(x)) (f(x) - g^{(D)}(x)) ]$ est $= 0$ 
	
	\begin{equation}
		\label{intermediate_proof_a_1}
		E_{y,D} [(y-f(x)) (f(x) - g^{(D)}(x)) ]
	\end{equation}
	
	On peut distribuer dans l'équation (\ref{intermediate_proof_a_1}), on obtient :
	
	\begin{equation}
		\label{intermediate_proof_a_2}
		E_{y,D} [yf(x) -yg^{(D)}(x) - f^2(x) + f(x)g^{(D)}(x)]
	\end{equation}
	
	En utilisant (\ref{decomposition_Biais_Variance}) dans l'équation (\ref{intermediate_proof_a_2}), on obtient :
	
	\begin{equation}
		\label{intermediate_proof_a_3}
		E_{y,D} [(f(x) + \epsilon)f(x) -(f(x) + \epsilon)g^{(D)}(x) - f^2(x) + f(x)g^{(D)}(x)]
	\end{equation}
	
	En séparant les différents éléments et en simplifiant dans (\ref{intermediate_proof_a_3}), on obtient :
	
	\begin{equation}
		\label{intermediate_proof_a_4}
		E_{y,D} [f^2(x) + \epsilon f(x)] - E_{y,D}[f(x)g^{(D)}(x) + \epsilon g^{(D)}(x)] - E_{y,D}[f^2(x)] + E_{y,D}[f(x)g^{(D)}(x)]
	\end{equation}
	
	En vérifiant les espérances, on peut encore simplifier l'équation (\ref{intermediate_proof_a_4}) en :
	
	\begin{equation}
		\label{intermediate_proof_a_5}
		f^2(x) + \epsilon f(x) - E_D[f(x)g^{(D)}(x) + \epsilon g^{(D)}(x)] - f^2(x) + E_D[ f(x)g^{(D)}(x)]
	\end{equation}
	
	En utilisant (\ref{g_moyen}) dans l'équation (\ref{intermediate_proof_a_5}), on obtient :
	
	\begin{equation}
		\label{intermediate_proof_a_6}
		f^2(x) + \epsilon f(x) - f(x)\bar{g}(x) + \epsilon g^{(D)}(x) - f^2(x) + f(x)\bar{g}(x)
	\end{equation}
	
	Pour faciliter la notation, nous avions fixé $x$, l'équation (\ref{intermediate_proof_a_6})  donne en réalité :
	
	\begin{equation}
		\label{intermediate_proof_a_7}
		E_x[f^2(x) + \epsilon f(x) - f(x)\bar{g}(x) + \epsilon g^{(D)}(x) - f^2(x) + f(x)\bar{g}(x)]
	\end{equation}
	
	En utilisant la définition de l'espérance de $\epsilon$ dans l'équation (\ref{intermediate_proof_a_7}), on obtient :
	
	\begin{equation}
		\label{intermediate_proof_a_8}
		E_x[f^2(x) + 0 f(x) - f(x)\bar{g}(x) + 0 g^{(D)}(x) - f^2(x) + f(x)\bar{g}(x)]
	\end{equation}
	
	En simplifiant l'équation (\ref{intermediate_proof_a_8}), on obtient finalement :
	
	\begin{equation}
		\label{intermediate_proof_a_9}
		E_x[f^2(x) - f(x)\bar{g}(x) - f^2(x) + f(x)\bar{g}(x)] = E_x[0] = 0
	\end{equation}
	
	On a donc prouvé mathématiquement que (\ref{intermediate_proof_a_1}) est bien égale à 0
	\newpage
	
	\section{Preuve intermédiaire de $2E_{y,D} [(f(x)-\bar{g}(x)) (\bar{g}(x) - g^{(D)}(x)) ]$}
	\label{intermediate_proof_b_subsection}
	Prouvons que $E_{y,D} [(f(x)-\bar{g}(x)) (\bar{g}(x) - g^{(D)}(x)) ]$ est $= 0$ 
	
	\begin{equation}
		\label{intermediate_proof_b_1}
		E_{y,D} [(f(x)-\bar{g}(x)) (\bar{g}(x) - g^{(D)}(x)) ]
	\end{equation}
	
	On peut distribuer dans l'équation (\ref{intermediate_proof_b_1}), on obtient :
	
	\begin{equation}
		\label{intermediate_proof_b_2}
		E_{y,D} [ f(x)\bar{g}(x) - f(x)g^{(D)}(x) -g^2(x) + \bar{g}(x)g^{(D)}(x)]
	\end{equation}
	
	en vérifiant les espérances dans l'équation (\ref{intermediate_proof_b_2}), on obtient :
	
	\begin{equation}
		\label{intermediate_proof_b_3}
		f(x)\bar{g}(x) + E_{D} [- f(x)g^{(D)}(x)] + E_{D} [ \bar{g}(x)g^{(D)}(x)] -g^2(x)
	\end{equation}
	
	en appliquant la formule du g moyen (\ref{g_moyen}), on obtient l'équation suivante :
	
	\begin{equation}
		\label{intermediate_proof_b_4}
		f(x)\bar{g}(x) - f(x)\bar{g}(x) + \bar{g^2}(x) -\bar{g^2}(x)
	\end{equation}
	
	Finalement, en simplifiant l'équation (\ref{intermediate_proof_b_4}), on obtient :
	
	\begin{equation}
		\label{intermediate_proof_b_5}
		f(x)\bar{g}(x) - f(x)\bar{g}(x) + \bar{g^2}(x) -\bar{g^2}(x) = 0
	\end{equation}
\end{appendices}

\newpage

\begin{thebibliography}{9}
	
	\bibitem{ReconcilingModernML}
	Belkin M., Hsu D., Ma S., Mandal S.,
	Reconciling modern machine learning practice and the bias-variance trade-off
	\textit{arXiv:1812.11118v2}, November 1-4, 2015, pp. 337-350.
	
	\bibitem{MLPracticalApproach}
	Fernandes de Mello R., Antonelli Ponti M.,
	\textit{Machine Learning A practical Approach on the Statistical Learning Theory},
	Springer, Cham, 2018.
	
	\bibitem{NeuralNetworksBiasVarianceDilemma}
	Geman S., Bienenstock E., Doursat R.,
	Neural Networks and the Bias/Variance Dilemma
	\textit{Neural Computation} 4, 1-58, 1992
	\url{http://direct.mit.edu/neco/article-pdf/4/1/1/812244/neco.1992.4.1.1.pdf}
	
	\bibitem{BiasVarianceTradeoffTextbooksUpdate}
	Neal B.,
	On the Bias-Variance Tradeoff : Textbooks Need an Update
	\textit{arXiv:1912.08286v1}, December 2019
	
	\bibitem{UnderstandingML}
	Shalev-Shwartz S., Ben-David S.,
	\textit{Understanding Machine Learning From Theory to Algorithms},
	Cambridge University Press, 2019 (12th printing).
	
	\bibitem{UnderstandingBiasVarianceTradeoff}
	\url{http://scott.fortmann-roe.com/docs/BiasVariance.html},
	consulté le 18 Juin 2021 à 09:25
	
	\bibitem{BiasVarianceTradeOffEliteDataScience}
	\url{https://elitedatascience.com/bias-variance-tradeoff},
	consulté le 21 Juin 2021 à 10:16
	
	\bibitem{ExperimentationBiasVariance}
	\url{https://github.com/sayanam/MachineLearning/blob/master/ExperimentationWithBiasAndVariance/BiasAndVariance_V2.ipynb},
	consulté le 15 Juillet 2021 à 18:20
	
	
\end{thebibliography}
\newpage

\listoffigures
\newpage


\end{document}